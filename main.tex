\documentclass[9pt]{developercv} % Default font size, values from 8-12pt are recommended
\begin{document}

%----------------------------------------------------------------------------------------


%----------------------------------------------------------------------------------------
%   TITLE AND CONTACT INFORMATION
%----------------------------------------------------------------------------------------

\begin{minipage}[t]{0.45\textwidth} % 45% of the page width for name
    \vspace{-\baselineskip} % Required for vertically aligning minipages

    % If your name is very short, use just one of the lines below
    % If your name is very long, reduce the font size or make the minipage wider and reduce the others proportionately
    \colorbox{black}{{\HUGE\textcolor{white}{\textbf{\MakeUppercase{Michael}}}}} % First name

    \colorbox{black}{{\HUGE\textcolor{white}{\textbf{\MakeUppercase{Sendker}}}}} % Last name

    \vspace{6pt}

    {\huge Software Developer} % Career or current job title
\end{minipage}
\begin{minipage}[t]{0.275\textwidth} % 27.5% of the page width for the first row of icons
    \vspace{-\baselineskip} % Required for vertically aligning minipages

    % The first parameter is the FontAwesome icon name, the second is the box size and the third is the text
    % Other icons can be found by referring to fontawesome.pdf (supplied with the template) and using the word after \fa in the command for the icon you want
    \icon{MapMarker}{12}{Tarpon Springs, FL}\\
    \icon{Phone}{12}{+1 (727) 272-3915}\\
    \icon{Envelope}{12}{\href{mailto:m@stdwtr.io}{m@stdwtr.io}}\\
    \icon{Linkedin}{12}{\href{https://linkedin.com/in/michael-sendker}{michael-sendker}}\\
\end{minipage}
\begin{minipage}[t]{0.275\textwidth} % 27.5% of the page width for the second row of icons
    \vspace{-\baselineskip} % Required for vertically aligning minipages

    % The first parameter is the FontAwesome icon name, the second is the box size and the third is the text
    % Other icons can be found by referring to fontawesome.pdf (supplied with the template) and using the word after \fa in the command for the icon you want
    \icon{Globe}{12}{\href{https://standingwater.io}{standingwater.io}}\\
    \icon{Globe}{12}{\href{https://blog.standingwater.io}{blog.standingwater.io}}\\
    \icon{Github}{12}{\href{https://github.com/malan88}{malan88}}\\
    \icon{StackOverflow}{12}{\href{https://stackoverflow.com/story/malan88}{malan88}}\\
\end{minipage}

% this vspace is largely useless and causes me to have trouble fitting everything
%\vspace{0.5cm}

%----------------------------------------------------------------------------------------
%   INTRODUCTION, SKILLS AND TECHNOLOGIES
%----------------------------------------------------------------------------------------

\cvsect{Who I Am}

%\begin{minipage}[t]{0.4\textwidth} % 40% of the page width for the introduction text
%   \vspace{-\baselineskip} % Required for vertically aligning minipages

    I prefer Python, but I wear many hats. I have written code in C\#,
    Python, vanilla JavaScript, and ReactJS. I have dabbled in C, Scheme, C++,
    and Bash. I have built static sites in GatsbyJS, interfaces in WPF, and web
    scrapers and bots in Python. I even built an arduino-based PID controller
    for a popcorn popper to roast my own coffee. But by far most of my time has
    been spent developing a single web technology, consisting of $\approx$
    14,000 lines of code in Python, Flask, and SQLAlchemy, in the pursuit of my
    obsession with literature:
    {\href{https://github.com/malan88/icc}{anno.wiki}}.
%\end{minipage}
%\hfill % Whitespace between
%\begin{minipage}[t]{0.5\textwidth} % 50% of the page for the skills bar chart
%   \vspace{-\baselineskip} % Required for vertically aligning minipages
%   \begin{barchart}{5.5}
%        \baritem{ {\href{https://www.python.org/}{Python}} }{100}
%        \baritem{ {\href{https://flask.palletsprojects.com/en/1.1.x/}{Flask}} }{100}
%        \baritem{ {\href{https://www.sqlalchemy.org/}{SQLAlchemy}} }{90}
%       \baritem{ {\href{https://en.wikipedia.org/wiki/JavaScript}{JavaScript}} }{80}
%       \baritem{ {\href{https://www.djangoproject.com/}{Django}} }{70}
%       \baritem{ {\href{https://reactjs.org/}{ReactJS}} }{80}
%   \end{barchart}
%\end{minipage}

%\begin{center}
%   \bubbles{
%        5/{\href{https://neovim.io/}{neovim}},
%        3/{\href{https://aws.amazon.com/}{AWS}},
%        4/{\href{https://www.heroku.com/}{Heroku}},
%        3/{\href{https://mariadb.org/}{MariaDB}},
%        3/{\href{https://www.gnu.org/software/bash/}{BASH}},
%        2/{\href{https://gnupg.org/}{GPG}},
%        5/{\href{https://sass-lang.com/}{Sass}},
%        5/{\href{https://www.linux.org/}{Linux}}
%    }
%\end{center}

%----------------------------------------------------------------------------------------
%   EXPERIENCE
%----------------------------------------------------------------------------------------

\cvsect{Experience}

\begin{entrylist}
    \entry
        {2017 -- 2019}
        {{\href{https://github.com/Anno-Wiki/icc}{Intertextual Canon Cloud (anno.wiki)}}}
        {Flask App}
        {The ICC is a web application designed to allow for collaboratively
        building an exhaustive and definitive repository of annotated
        literature. I designed, developed, deployed, and continue to maintain
        the project solo, managing occasionally to rope in some programmer
        friends for help with various features. It consists of $\approx$ 14k
        lines of code and $\approx$ 100k lines of code churn. The backend uses
        Flask/SQLAlchemy. The frontend uses Jinja2, Sass, and VanillaJS. I also
        had to write several ETL data pipelines for processing
        {\href{https://gutenberg.org}{Project Gutenberg}} texts. It is deployed
        via Heroku. I learned the entire web application life cycle on this
        project, and it continues to teach me. It is currently maintained at
        {\href{https://anno.wiki}{https://anno.wiki}}.
        \\
        \texttt{{\href{https://www.python.org/}{Python}}}\slashsep
        \texttt{{\href{https://flask.palletsprojects.com/en/1.1.x/}{Flask}}}\slashsep
        \texttt{{\href{https://www.sqlalchemy.org/}{SQLAlchemy}}}\slashsep
        \texttt{{\href{https://mariadb.org/}{MariaDB}}}\slashsep
        \texttt{{\href{https://www.elastic.co/}{Elasticsearch}}}\slashsep
        \texttt{{\href{https://docs.pytest.org/en/stable/}{pytest}}}\slashsep
        \texttt{{\href{https://www.heroku.com/}{Heroku}}}
        }
    \entry
        {2019 -- Current}
        {{\href{https://github.com/Anno-Wiki}{Intertextual Canon Cloud 2}}}
        {Flask/React App}
        {
            This is the second iteration of the ICC, with improved data
            architecture and a single-page application front end based in React.
            Using Elasticsearch to store texts and PostgreSQL to store
            annotations and general application data allows for faster
            incremental searching. I also wrote more sophisticated text
            processors for breaking text into 100k byte chunks and annotating
            stylistic elements, while stripping all styling from the raw text.
        \\
        \texttt{{\href{https://www.python.org/}{Python}}}\slashsep
        \texttt{{\href{https://flask.palletsprojects.com/en/1.1.x/}{Flask}}}\slashsep
        \texttt{{\href{https://www.sqlalchemy.org/}{SQLAlchemy}}}\slashsep
        \texttt{{\href{https://www.elastic.co/}{Elasticsearch}}}\slashsep
        \texttt{{\href{https://www.postgresql.org/}{PostgreSQL}}}\slashsep
        \texttt{{\href{https://www.docker.com/}{Docker}}}\slashsep
        \texttt{{\href{https://auth0.com/}{Auth0}}}\slashsep
        \texttt{{\href{https://reactjs.org/}{ReactJS}}}
        }
    \entry
        {2020 -- Current}
        {\href{https://euler-sci.com}{Euler Sciences LLC}}
        {Contract Developer}
        {I worked on a GUI interface for laser system meant to treat skin
        conditions. I also developed the company website in GatsbyJS (and did
        the graphical work myself in Photoshop).
        \\
        \texttt{{\href{https://en.wikipedia.org/wiki/C_Sharp_(programming_language)}{C\#}}}\slashsep
        \texttt{{\href{https://en.wikipedia.org/wiki/Windows_Presentation_Foundation}{Windows Presentation Foundation}}}\slashsep
        \texttt{{\href{https://www.sqlite.org/index.html}{SQLite}}}\slashsep
        \texttt{{\href{https://www.adobe.com/products/photoshop.html}{Photoshop}}}\slashsep
        \texttt{{\href{https://www.gatsbyjs.org/}{GatsbyJS}}}
        }
    \entry
        {2020 -- Current}
        {Healthy Brands, LLC}
        {Contract Developer}
        {
            I've worked on several projects for this company: a multithreaded
            scraper and alert system for notifying the company of BuyBox loss
            events on Amazon, I launched and modified a Django-based question
            and answer forum from biostars, prepared and maintained several VPS
            servers, one in Arch Linux, designed and maintained several
            WordPress sites, integrated Twilio into the company's ZOHO CRM
            software, and modified several Shopify templates.
        \\
        \texttt{{\href{https://www.python.org/}{Python}}}\slashsep
        \texttt{{\href{https://www.crummy.com/software/BeautifulSoup/bs4/doc/}{BeautifulSoup4}}}\slashsep
        \texttt{{\href{https://www.zoho.com/}{Zoho}}}\slashsep
        \texttt{{\href{https://www.twilio.com/}{Twilio}}}\slashsep
        \texttt{{\href{https://wordpress.org/}{WordPress}}}\slashsep
        \texttt{{\href{https://www.shopify.com/}{Shopify}}}
        }
    \entry
		{2019}
		{Minimal Pairs Scraping Project}
		{Scraper}
        {I generated an Anki deck for recognizing French phonemes difficult to
        hear for an American English speaker using “minimal pairs,” words in
        French which differ by only one phoneme. Limit for requests was 500 per
        day, had to save progress and download the data over several days.
        \\
        \texttt{Python}\slashsep
        \texttt{requests}\slashsep
        \texttt{BeautifulSoup4}\slashsep
        \texttt{json}\slashsep
        \texttt{csv}
        }
\end{entrylist}

%----------------------------------------------------------------------------------------
%   EDUCATION
%----------------------------------------------------------------------------------------

\cvsect{Education}

\begin{entrylist}
    \entry
        {2008 -- 2010}
        {Philosophy and Classics}
        {Florida State University}
        {Focus on Ancient Philosophy, particularly Aristotle, Philosphy of Mind,
        and Symbolic Logic. Ancient Greek and Latin. President of Philosophy
        Club (2010)}
    \entry
        {2007 -- 2008}
        {Philosophy}
        {University of South Florida}
        {General introductory classes, Social Philosophy, Critical Thinking}
\end{entrylist}

%----------------------------------------------------------------------------------------
%   ADDITIONAL INFORMATION
%----------------------------------------------------------------------------------------

\begin{minipage}[t]{0.2\textwidth}
    \vspace{-\baselineskip} % Required for vertically aligning minipages

    \cvsect{Languages}

        \textbf{English} — native,\\
        \textbf{French} — fluent,\\
        \textbf{Spanish} — rudimentary

\end{minipage}
\hfill
\begin{minipage}[t]{0.3\textwidth}
    \vspace{-\baselineskip} % Required for vertically aligning minipages

    \cvsect{Free Time}

    I love high culture. Coffee, wine, cheese, cocktails, literature, music,
    art, cinema, science; anything that makes life worth living. And running.
\end{minipage}
\hfill
\begin{minipage}[t]{0.35\textwidth}
    \vspace{-\baselineskip} % Required for vertically aligning minipages

    \cvsect{Closing Argument}

    I have had enough experience learning new things to have full confidence in
    my ability to learn new things.
\end{minipage}

%----------------------------------------------------------------------------------------

\end{document}
